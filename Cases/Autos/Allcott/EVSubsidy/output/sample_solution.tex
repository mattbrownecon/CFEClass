% Options for packages loaded elsewhere
\PassOptionsToPackage{unicode}{hyperref}
\PassOptionsToPackage{hyphens}{url}
\PassOptionsToPackage{dvipsnames,svgnames*,x11names*}{xcolor}
%
\documentclass[
]{article}
\usepackage{amsmath,amssymb}
\usepackage{lmodern}
\usepackage{iftex}
\ifPDFTeX
  \usepackage[T1]{fontenc}
  \usepackage[utf8]{inputenc}
  \usepackage{textcomp} % provide euro and other symbols
\else % if luatex or xetex
  \usepackage{unicode-math}
  \defaultfontfeatures{Scale=MatchLowercase}
  \defaultfontfeatures[\rmfamily]{Ligatures=TeX,Scale=1}
\fi
% Use upquote if available, for straight quotes in verbatim environments
\IfFileExists{upquote.sty}{\usepackage{upquote}}{}
\IfFileExists{microtype.sty}{% use microtype if available
  \usepackage[]{microtype}
  \UseMicrotypeSet[protrusion]{basicmath} % disable protrusion for tt fonts
}{}
\makeatletter
\@ifundefined{KOMAClassName}{% if non-KOMA class
  \IfFileExists{parskip.sty}{%
    \usepackage{parskip}
  }{% else
    \setlength{\parindent}{0pt}
    \setlength{\parskip}{6pt plus 2pt minus 1pt}}
}{% if KOMA class
  \KOMAoptions{parskip=half}}
\makeatother
\usepackage{xcolor}
\IfFileExists{xurl.sty}{\usepackage{xurl}}{} % add URL line breaks if available
\IfFileExists{bookmark.sty}{\usepackage{bookmark}}{\usepackage{hyperref}}
\hypersetup{
  pdftitle={Problem Set 5 Sample Solutions},
  pdfauthor={Matt Brown},
  colorlinks=true,
  linkcolor={Maroon},
  filecolor={Maroon},
  citecolor={Blue},
  urlcolor={blue},
  pdfcreator={LaTeX via pandoc}}
\urlstyle{same} % disable monospaced font for URLs
\usepackage[margin=1in]{geometry}
\usepackage{color}
\usepackage{fancyvrb}
\newcommand{\VerbBar}{|}
\newcommand{\VERB}{\Verb[commandchars=\\\{\}]}
\DefineVerbatimEnvironment{Highlighting}{Verbatim}{commandchars=\\\{\}}
% Add ',fontsize=\small' for more characters per line
\usepackage{framed}
\definecolor{shadecolor}{RGB}{248,248,248}
\newenvironment{Shaded}{\begin{snugshade}}{\end{snugshade}}
\newcommand{\AlertTok}[1]{\textcolor[rgb]{0.94,0.16,0.16}{#1}}
\newcommand{\AnnotationTok}[1]{\textcolor[rgb]{0.56,0.35,0.01}{\textbf{\textit{#1}}}}
\newcommand{\AttributeTok}[1]{\textcolor[rgb]{0.77,0.63,0.00}{#1}}
\newcommand{\BaseNTok}[1]{\textcolor[rgb]{0.00,0.00,0.81}{#1}}
\newcommand{\BuiltInTok}[1]{#1}
\newcommand{\CharTok}[1]{\textcolor[rgb]{0.31,0.60,0.02}{#1}}
\newcommand{\CommentTok}[1]{\textcolor[rgb]{0.56,0.35,0.01}{\textit{#1}}}
\newcommand{\CommentVarTok}[1]{\textcolor[rgb]{0.56,0.35,0.01}{\textbf{\textit{#1}}}}
\newcommand{\ConstantTok}[1]{\textcolor[rgb]{0.00,0.00,0.00}{#1}}
\newcommand{\ControlFlowTok}[1]{\textcolor[rgb]{0.13,0.29,0.53}{\textbf{#1}}}
\newcommand{\DataTypeTok}[1]{\textcolor[rgb]{0.13,0.29,0.53}{#1}}
\newcommand{\DecValTok}[1]{\textcolor[rgb]{0.00,0.00,0.81}{#1}}
\newcommand{\DocumentationTok}[1]{\textcolor[rgb]{0.56,0.35,0.01}{\textbf{\textit{#1}}}}
\newcommand{\ErrorTok}[1]{\textcolor[rgb]{0.64,0.00,0.00}{\textbf{#1}}}
\newcommand{\ExtensionTok}[1]{#1}
\newcommand{\FloatTok}[1]{\textcolor[rgb]{0.00,0.00,0.81}{#1}}
\newcommand{\FunctionTok}[1]{\textcolor[rgb]{0.00,0.00,0.00}{#1}}
\newcommand{\ImportTok}[1]{#1}
\newcommand{\InformationTok}[1]{\textcolor[rgb]{0.56,0.35,0.01}{\textbf{\textit{#1}}}}
\newcommand{\KeywordTok}[1]{\textcolor[rgb]{0.13,0.29,0.53}{\textbf{#1}}}
\newcommand{\NormalTok}[1]{#1}
\newcommand{\OperatorTok}[1]{\textcolor[rgb]{0.81,0.36,0.00}{\textbf{#1}}}
\newcommand{\OtherTok}[1]{\textcolor[rgb]{0.56,0.35,0.01}{#1}}
\newcommand{\PreprocessorTok}[1]{\textcolor[rgb]{0.56,0.35,0.01}{\textit{#1}}}
\newcommand{\RegionMarkerTok}[1]{#1}
\newcommand{\SpecialCharTok}[1]{\textcolor[rgb]{0.00,0.00,0.00}{#1}}
\newcommand{\SpecialStringTok}[1]{\textcolor[rgb]{0.31,0.60,0.02}{#1}}
\newcommand{\StringTok}[1]{\textcolor[rgb]{0.31,0.60,0.02}{#1}}
\newcommand{\VariableTok}[1]{\textcolor[rgb]{0.00,0.00,0.00}{#1}}
\newcommand{\VerbatimStringTok}[1]{\textcolor[rgb]{0.31,0.60,0.02}{#1}}
\newcommand{\WarningTok}[1]{\textcolor[rgb]{0.56,0.35,0.01}{\textbf{\textit{#1}}}}
\usepackage{graphicx}
\makeatletter
\def\maxwidth{\ifdim\Gin@nat@width>\linewidth\linewidth\else\Gin@nat@width\fi}
\def\maxheight{\ifdim\Gin@nat@height>\textheight\textheight\else\Gin@nat@height\fi}
\makeatother
% Scale images if necessary, so that they will not overflow the page
% margins by default, and it is still possible to overwrite the defaults
% using explicit options in \includegraphics[width, height, ...]{}
\setkeys{Gin}{width=\maxwidth,height=\maxheight,keepaspectratio}
% Set default figure placement to htbp
\makeatletter
\def\fps@figure{htbp}
\makeatother
\setlength{\emergencystretch}{3em} % prevent overfull lines
\providecommand{\tightlist}{%
  \setlength{\itemsep}{0pt}\setlength{\parskip}{0pt}}
\setcounter{secnumdepth}{-\maxdimen} % remove section numbering
\usepackage{booktabs}
\usepackage{longtable}
\usepackage{array}
\usepackage{multirow}
\usepackage{wrapfig}
\usepackage{float}
\usepackage{colortbl}
\usepackage{pdflscape}
\usepackage{tabu}
\usepackage{threeparttable}
\usepackage{threeparttablex}
\usepackage[normalem]{ulem}
\usepackage{makecell}
\usepackage{xcolor}
\usepackage{siunitx}

  \newcolumntype{d}{S[
    input-open-uncertainty=,
    input-close-uncertainty=,
    parse-numbers = false,
    table-align-text-pre=false,
    table-align-text-post=false
  ]}
  
\ifLuaTeX
  \usepackage{selnolig}  % disable illegal ligatures
\fi

\title{Problem Set 5 Sample Solutions}
\author{Matt Brown}
\date{2023-05-10}

\begin{document}
\maketitle

\hypertarget{part-1}{%
\section{Part 1}\label{part-1}}

\hypertarget{a}{%
\subsubsection{1.A}\label{a}}

\begin{Shaded}
\begin{Highlighting}[]
\FunctionTok{ggplot}\NormalTok{(}\AttributeTok{data =}\NormalTok{ df\_EV, }\FunctionTok{aes}\NormalTok{(}\AttributeTok{x =}\NormalTok{ price, }\AttributeTok{y =}\NormalTok{ quantity)) }\SpecialCharTok{+} 
  \FunctionTok{geom\_point}\NormalTok{() }\SpecialCharTok{+}
  \FunctionTok{geom\_smooth}\NormalTok{(}\AttributeTok{method =} \StringTok{\textquotesingle{}lm\textquotesingle{}}\NormalTok{, }\AttributeTok{color =} \StringTok{\textquotesingle{}grey\textquotesingle{}}\NormalTok{, }\AttributeTok{se =} \ConstantTok{FALSE}\NormalTok{, }\AttributeTok{linetype =} \StringTok{\textquotesingle{}dashed\textquotesingle{}}\NormalTok{) }\SpecialCharTok{+} 
  \FunctionTok{geom\_hline}\NormalTok{(}\FunctionTok{aes}\NormalTok{(}\AttributeTok{yintercept =} \DecValTok{0}\NormalTok{)) }\SpecialCharTok{+} \FunctionTok{geom\_vline}\NormalTok{(}\FunctionTok{aes}\NormalTok{(}\AttributeTok{xintercept =} \DecValTok{0}\NormalTok{)) }\SpecialCharTok{+}
  \FunctionTok{labs}\NormalTok{(}\AttributeTok{x =} \StringTok{\textquotesingle{}Price ($000s)\textquotesingle{}}\NormalTok{, }\AttributeTok{y =} \StringTok{\textquotesingle{}Quantity (000s)\textquotesingle{}}\NormalTok{)}
\end{Highlighting}
\end{Shaded}

\includegraphics{../output/sample_solution_files/figure-latex/1.A-1.pdf}

\hypertarget{b}{%
\subsubsection{1.B}\label{b}}

This relationship is not plausibly causal. The plot shows that expensive
vehicles are purchased less, which is consistent with the standard
intuition of a (causal) downward sloping demand curve. However,
confounders could also create a relationship between price and quantity.
For example, if higher quality vehicles are also more expensive (seems
plausible!) then there would be a spurious \emph{positive} relationship
between price and quantity. If quality is an omitted variable, then the
true causal effect of price on quantity is steeper than the best fit
line in this plot.

\hypertarget{c-and-1.d}{%
\subsubsection{1.C and 1.D}\label{c-and-1.d}}

\begin{Shaded}
\begin{Highlighting}[]
\NormalTok{scc }\OtherTok{=} \DecValTok{190} \CommentTok{\# social cost of carbon}
\NormalTok{df\_EV }\OtherTok{=}\NormalTok{ df\_EV }\SpecialCharTok{\%\textgreater{}\%} \FunctionTok{mutate}\NormalTok{(}\AttributeTok{co2cost =}\NormalTok{ co2}\SpecialCharTok{*}\NormalTok{scc)}

\NormalTok{avg\_EV\_co2cost }\OtherTok{=} \FunctionTok{mean}\NormalTok{(df\_EV[df\_EV}\SpecialCharTok{$}\NormalTok{ev}\SpecialCharTok{==}\DecValTok{1}\NormalTok{,]}\SpecialCharTok{$}\NormalTok{co2cost)}
\NormalTok{avg\_nonEV\_co2cost }\OtherTok{=} \FunctionTok{mean}\NormalTok{(df\_EV[df\_EV}\SpecialCharTok{$}\NormalTok{ev}\SpecialCharTok{==}\DecValTok{0}\NormalTok{,]}\SpecialCharTok{$}\NormalTok{co2cost)}

\FunctionTok{ggplot}\NormalTok{(}\AttributeTok{data =}\NormalTok{ df\_EV }\SpecialCharTok{\%\textgreater{}\%} \FunctionTok{filter}\NormalTok{(}\SpecialCharTok{!}\NormalTok{ev, price}\SpecialCharTok{\textless{}}\DecValTok{50}\NormalTok{), }\FunctionTok{aes}\NormalTok{(}\AttributeTok{x =}\NormalTok{ co2cost)) }\SpecialCharTok{+}
  \FunctionTok{geom\_histogram}\NormalTok{(}\AttributeTok{bins =} \DecValTok{15}\NormalTok{) }\SpecialCharTok{+} 
  \FunctionTok{geom\_vline}\NormalTok{(}\FunctionTok{aes}\NormalTok{(}\AttributeTok{xintercept =}\NormalTok{ avg\_EV\_co2cost), }\AttributeTok{color =} \StringTok{\textquotesingle{}blue\textquotesingle{}}\NormalTok{, }\AttributeTok{linetype =} \StringTok{\textquotesingle{}dashed\textquotesingle{}}\NormalTok{) }\SpecialCharTok{+}
  \FunctionTok{labs}\NormalTok{(}\AttributeTok{x =} \StringTok{\textquotesingle{}Lifetime CO2 damages (dollars)\textquotesingle{}}\NormalTok{)}
\end{Highlighting}
\end{Shaded}

\includegraphics{../output/sample_solution_files/figure-latex/1.Cand1.D-1.pdf}

The average CO2 damages for EVs is \$3237 (blue dotted line on graph).
The average cost for non-EVs is \$\ensuremath{1.3809\times 10^{4}}. To
be fair, the non-EV average is misleading since two outliers with very
low market share (Rolls Royce Phantom and Lamborghini Aventador) also
have very high CO2 damages. Nevertheless, it's clear fro the histogram
that 1) all non-EVs have higher CO2 costs than EVs and 2) even among
mass-market EVs there is significant heterogeneity in CO2 costs.

\hypertarget{part-2}{%
\section{Part 2}\label{part-2}}

\begin{Shaded}
\begin{Highlighting}[]
\CommentTok{\#define market shares}
\NormalTok{market\_size }\OtherTok{=} \DecValTok{250000000}
\NormalTok{df\_EV }\OtherTok{=}\NormalTok{ df\_EV }\SpecialCharTok{\%\textgreater{}\%} \FunctionTok{mutate}\NormalTok{(}\AttributeTok{share =}\NormalTok{ quantity }\SpecialCharTok{*} \DecValTok{1000} \SpecialCharTok{/}\NormalTok{ market\_size)}
\CommentTok{\# define share of outside option}
\NormalTok{share\_buy }\OtherTok{=} \FunctionTok{sum}\NormalTok{(df\_EV}\SpecialCharTok{$}\NormalTok{share)}
\NormalTok{s0 }\OtherTok{=} \DecValTok{1}\SpecialCharTok{{-}}\NormalTok{share\_buy}
\CommentTok{\# define DV}
\NormalTok{df\_EV }\OtherTok{=}\NormalTok{ df\_EV }\SpecialCharTok{\%\textgreater{}\%} \FunctionTok{mutate}\NormalTok{(}\AttributeTok{lns1s0 =} \FunctionTok{log}\NormalTok{(share}\SpecialCharTok{/}\NormalTok{s0))}

\CommentTok{\# cosmetic changes for the table}
\NormalTok{df\_EV}\SpecialCharTok{$}\NormalTok{class }\OtherTok{=} \FunctionTok{factor}\NormalTok{(df\_EV}\SpecialCharTok{$}\NormalTok{class)}
\NormalTok{df\_EV}\SpecialCharTok{$}\NormalTok{class }\OtherTok{=} \FunctionTok{ref}\NormalTok{(df\_EV}\SpecialCharTok{$}\NormalTok{class, }\StringTok{\textquotesingle{}sedan\textquotesingle{}}\NormalTok{)}
\NormalTok{df\_EV}\SpecialCharTok{$}\NormalTok{fuel\_cost\_thousands }\OtherTok{=}\NormalTok{ df\_EV}\SpecialCharTok{$}\NormalTok{fuel\_cost}\SpecialCharTok{/}\DecValTok{1000}

\NormalTok{logit\_out }\OtherTok{=} \FunctionTok{lm\_robust}\NormalTok{(}\AttributeTok{data =}\NormalTok{ df\_EV,}
                      \AttributeTok{formula =}\NormalTok{ lns1s0 }\SpecialCharTok{\textasciitilde{}}\NormalTok{ price }\SpecialCharTok{+}\NormalTok{ weight }\SpecialCharTok{+}\NormalTok{ fuel\_cost }\SpecialCharTok{+}\NormalTok{ hp }\SpecialCharTok{+}\NormalTok{ ev)}

\NormalTok{Tab1 }\OtherTok{=} \FunctionTok{modelsummary}\NormalTok{(logit\_out,}
             \AttributeTok{coef\_map =}\NormalTok{ coef\_map,}
             \AttributeTok{notes =} \StringTok{\textquotesingle{}The leave{-}out group is sedan\textquotesingle{}}\NormalTok{,}
             \AttributeTok{title =} \StringTok{\textquotesingle{}Logit Results\textquotesingle{}}\NormalTok{,}
             \AttributeTok{fmt =} \FunctionTok{fmt\_significant}\NormalTok{(}\DecValTok{2}\NormalTok{))}
\NormalTok{Tab1}
\end{Highlighting}
\end{Shaded}

\begin{table}

\caption{\label{tab:2.A}Logit Results}
\centering
\begin{tabular}[t]{lc}
\toprule
  & (1)\\
\midrule
Price (000s) & \num{-0.0251}\\
 & (\num{0.0053})\\
Electric Vehicle & \num{1.3}\\
 & (\num{1.1})\\
Weight (tons) & \num{-1.83}\\
 & (\num{0.71})\\
Horsepower (00s) & \num{0.49}\\
 & (\num{0.21})\\
Constant & \num{-7.11}\\
 & (\num{0.84})\\
\midrule
Num.Obs. & \num{35}\\
R2 & \num{0.797}\\
R2 Adj. & \num{0.762}\\
AIC & \num{94.4}\\
BIC & \num{105.3}\\
RMSE & \num{0.76}\\
\bottomrule
\multicolumn{2}{l}{\rule{0pt}{1em}The leave-out group is sedan}\\
\end{tabular}
\end{table}

\begin{Shaded}
\begin{Highlighting}[]
\NormalTok{price\_coef }\OtherTok{=}\NormalTok{ logit\_out}\SpecialCharTok{$}\NormalTok{coefficients[}\StringTok{\textquotesingle{}price\textquotesingle{}}\NormalTok{]}
\NormalTok{ev\_coef }\OtherTok{=}\NormalTok{ logit\_out}\SpecialCharTok{$}\NormalTok{coefficients[}\StringTok{\textquotesingle{}ev\textquotesingle{}}\NormalTok{]}
\end{Highlighting}
\end{Shaded}

\hypertarget{b-1}{%
\subsubsection{2.B}\label{b-1}}

The results indicate a thousand dollar price increase is associated with
a -2.5\% change in market share, and that EVs have 125.1\% higher market
share than non-EVs, all else equal. The EV coefficient is not
statistically significant, but the price effect is.

\hypertarget{c}{%
\subsubsection{2.C}\label{c}}

I think the estimate of \(\eta\) is biased towards zero. Unobservable
quality is likely positively correlated with both price and demand,
which will cause an upwards bias in our coefficient estimate (in this
case, pushing towards zero). Since the set of observables is not
\emph{that} rich here, omitted variable bias could be large in
magnitude.

\hypertarget{part-3}{%
\section{Part 3}\label{part-3}}

\hypertarget{a-1}{%
\subsubsection{3.A}\label{a-1}}

Policy incidence falls on the less elastic group. We're assuming
perfectly elastic supply. Therefore, all of the incidence will fall on
consumers.

\hypertarget{b-2}{%
\subsubsection{3.B}\label{b-2}}

\begin{Shaded}
\begin{Highlighting}[]
\NormalTok{avg\_CO2\_tax }\OtherTok{=} \FunctionTok{mean}\NormalTok{(df\_EV}\SpecialCharTok{$}\NormalTok{co2cost)}
\end{Highlighting}
\end{Shaded}

The cross-vehicle average CO2 tax is \$\ensuremath{1.0789\times 10^{4}}.

\hypertarget{c-1}{%
\subsubsection{3.C}\label{c-1}}

In this model, the car's value is
\(V_j = \eta p_j + \beta X_j + \xi_j\). We want to hold everything
constant except the a price change, and see how \(V\) changes. Note that
\(V^{New} - V^{Baseline} = \beta (p^{New} - p^{Baseline})\). I use this
identity to compute the new Vs. Then, I can back out the market shares
according to the following formula:

\[
s_j = \frac{e^{V_j}}{1+\sum_k e^{V_k}}
\]

\begin{Shaded}
\begin{Highlighting}[]
\CommentTok{\# Step 1: create new price variables (taking care for units!)}
\NormalTok{cvc\_subsidy }\OtherTok{=} \FloatTok{7.5}
\NormalTok{df\_EV }\OtherTok{=}\NormalTok{ df\_EV }\SpecialCharTok{\%\textgreater{}\%}
  \FunctionTok{mutate}\NormalTok{(}\AttributeTok{p1 =}\NormalTok{ price }\SpecialCharTok{+}\NormalTok{ co2cost}\SpecialCharTok{/}\DecValTok{1000}\NormalTok{, }
         \AttributeTok{p2 =}\NormalTok{ price }\SpecialCharTok{{-}}\NormalTok{ eligible }\SpecialCharTok{*}\NormalTok{ cvc\_subsidy)}

\CommentTok{\# Step 2: create change in price variables}
\NormalTok{df\_EV }\OtherTok{=}\NormalTok{ df\_EV }\SpecialCharTok{\%\textgreater{}\%}
  \FunctionTok{mutate}\NormalTok{(}\AttributeTok{del\_p1 =}\NormalTok{p1 }\SpecialCharTok{{-}}\NormalTok{ price, }
         \AttributeTok{del\_p2 =}\NormalTok{p2 }\SpecialCharTok{{-}}\NormalTok{ price)}

\CommentTok{\# step 3: define V0 from baseline shares}
\CommentTok{\# (note that this line is redundant but the new variable name makes code easier to read)}
\NormalTok{df\_EV}\SpecialCharTok{$}\NormalTok{V0 }\OtherTok{=}\NormalTok{ df\_EV}\SpecialCharTok{$}\NormalTok{lns1s0}

\CommentTok{\# step 4: compute new "V"s}
\NormalTok{df\_EV}\SpecialCharTok{$}\NormalTok{V1 }\OtherTok{=}\NormalTok{ df\_EV}\SpecialCharTok{$}\NormalTok{V0 }\SpecialCharTok{+}\NormalTok{ logit\_out}\SpecialCharTok{$}\NormalTok{coefficients[}\StringTok{\textquotesingle{}price\textquotesingle{}}\NormalTok{] }\SpecialCharTok{*}\NormalTok{ df\_EV}\SpecialCharTok{$}\NormalTok{del\_p1}
\NormalTok{df\_EV}\SpecialCharTok{$}\NormalTok{V2 }\OtherTok{=}\NormalTok{ df\_EV}\SpecialCharTok{$}\NormalTok{V0 }\SpecialCharTok{+}\NormalTok{ logit\_out}\SpecialCharTok{$}\NormalTok{coefficients[}\StringTok{\textquotesingle{}price\textquotesingle{}}\NormalTok{] }\SpecialCharTok{*}\NormalTok{ df\_EV}\SpecialCharTok{$}\NormalTok{del\_p2}

\FunctionTok{cat}\NormalTok{(}\StringTok{\textquotesingle{}Avg V at baseline: \textquotesingle{}}\NormalTok{, }\FunctionTok{mean}\NormalTok{(df\_EV}\SpecialCharTok{$}\NormalTok{V0), }\StringTok{\textquotesingle{}}\SpecialCharTok{\textbackslash{}n}\StringTok{\textquotesingle{}}\NormalTok{)}
\end{Highlighting}
\end{Shaded}

\begin{verbatim}
## Avg V at baseline:  -9.114336
\end{verbatim}

\begin{Shaded}
\begin{Highlighting}[]
\FunctionTok{cat}\NormalTok{(}\StringTok{\textquotesingle{}Avg V with tax: \textquotesingle{}}\NormalTok{, }\FunctionTok{mean}\NormalTok{(df\_EV}\SpecialCharTok{$}\NormalTok{V1), }\StringTok{\textquotesingle{}}\SpecialCharTok{\textbackslash{}n}\StringTok{\textquotesingle{}}\NormalTok{)}
\end{Highlighting}
\end{Shaded}

\begin{verbatim}
## Avg V with tax:  -9.385245
\end{verbatim}

\begin{Shaded}
\begin{Highlighting}[]
\FunctionTok{cat}\NormalTok{(}\StringTok{\textquotesingle{}Avg V with subsidy: \textquotesingle{}}\NormalTok{, }\FunctionTok{mean}\NormalTok{(df\_EV}\SpecialCharTok{$}\NormalTok{V2), }\StringTok{\textquotesingle{}}\SpecialCharTok{\textbackslash{}n}\StringTok{\textquotesingle{}}\NormalTok{)}
\end{Highlighting}
\end{Shaded}

\begin{verbatim}
## Avg V with subsidy:  -9.071289
\end{verbatim}

\begin{Shaded}
\begin{Highlighting}[]
\CommentTok{\# step 5: compute key term for denominator of share definition}
\NormalTok{sum\_exp\_V1 }\OtherTok{=} \FunctionTok{sum}\NormalTok{(}\FunctionTok{exp}\NormalTok{(df\_EV}\SpecialCharTok{$}\NormalTok{V1))}
\NormalTok{sum\_exp\_V2 }\OtherTok{=} \FunctionTok{sum}\NormalTok{(}\FunctionTok{exp}\NormalTok{(df\_EV}\SpecialCharTok{$}\NormalTok{V2))}


\CommentTok{\# step 6: counterfactual market shares}
\NormalTok{df\_EV }\OtherTok{=}\NormalTok{ df\_EV }\SpecialCharTok{\%\textgreater{}\%}
  \FunctionTok{mutate}\NormalTok{(}
    \AttributeTok{s1 =} \FunctionTok{exp}\NormalTok{(V1)}\SpecialCharTok{/}\NormalTok{(}\DecValTok{1}\SpecialCharTok{+}\NormalTok{sum\_exp\_V1),}
    \AttributeTok{s2 =} \FunctionTok{exp}\NormalTok{(V2)}\SpecialCharTok{/}\NormalTok{(}\DecValTok{1}\SpecialCharTok{+}\NormalTok{sum\_exp\_V2)}
\NormalTok{    )}

\CommentTok{\# Step 7: illustrate with a few vehicles }
\NormalTok{df\_illustrate }\OtherTok{=}\NormalTok{ df\_EV }\SpecialCharTok{\%\textgreater{}\%} 
  \FunctionTok{filter}\NormalTok{(model }\SpecialCharTok{==} \StringTok{\textquotesingle{}model y\textquotesingle{}} \SpecialCharTok{|}\NormalTok{ model }\SpecialCharTok{==} \StringTok{\textquotesingle{}civic\textquotesingle{}} \SpecialCharTok{|}\NormalTok{ model }\SpecialCharTok{==} \StringTok{\textquotesingle{}tacoma\textquotesingle{}}\NormalTok{) }\SpecialCharTok{\%\textgreater{}\%}
  \FunctionTok{select}\NormalTok{(make, model, co2cost, share, s1, s2) }\SpecialCharTok{\%\textgreater{}\%}
  \FunctionTok{arrange}\NormalTok{(co2cost)}

\CommentTok{\# total vehicle purchases}
\NormalTok{sum\_share\_baseline }\OtherTok{=} \FunctionTok{sum}\NormalTok{(df\_EV}\SpecialCharTok{$}\NormalTok{share)}
\NormalTok{sum\_share\_1 }\OtherTok{=} \FunctionTok{sum}\NormalTok{(df\_EV}\SpecialCharTok{$}\NormalTok{s1)}
\NormalTok{sum\_share\_2 }\OtherTok{=} \FunctionTok{sum}\NormalTok{(df\_EV}\SpecialCharTok{$}\NormalTok{s2)}
\end{Highlighting}
\end{Shaded}

In the baseline scenario 0.74\% of potential consumers buy a car. In the
carbon tax scenario, this number decreases to 0.58\%, while in the
subsidy scenario it increases to 0.78\% (these numbers are the inverse
of the share for the outside option). Since taxes raise the price of
vehicles on average, while subsidies decrease the price of vehicles on
average, the direction of these effects make sense.

\hypertarget{d}{%
\subsubsection{3.D}\label{d}}

The code for 3.D is given in the 3.C block.

In Table 2 I present market results for three specific vehicles that
illustrate the main effects. The carbon tax decrease consumption of all
car types, but the percent reduction in market share is larger for more
carbon-intensive vehicles (Toyota Tacoma) and smaller for less carbon
intensive vehicles (Tesla). The EV subsidy increases consumption of the
Tesla, while having only a minor effect on consumption of the two gas
vehicles. Importantly, the EV subsidy treats the low-emissions gas car
(Honda Civic) the same way it treats the Tacoma, which demonstrates that
the policy does not perfectly target emissions reduction.

\begin{Shaded}
\begin{Highlighting}[]
\NormalTok{kableExtra}\SpecialCharTok{::}\FunctionTok{kable}\NormalTok{(df\_illustrate, }\AttributeTok{booktabs =}\NormalTok{ T, }\AttributeTok{caption =} \StringTok{\textquotesingle{}Illustration of Policy Effects for Three Vehicles\textquotesingle{}}\NormalTok{,}
                  \AttributeTok{col.names =} \FunctionTok{c}\NormalTok{(}\StringTok{\textquotesingle{}Make\textquotesingle{}}\NormalTok{, }\StringTok{\textquotesingle{}Model\textquotesingle{}}\NormalTok{, }\StringTok{\textquotesingle{}Lifetime CO2 costs\textquotesingle{}}\NormalTok{,}
                                \StringTok{\textquotesingle{}Baseline share\textquotesingle{}}\NormalTok{, }\StringTok{\textquotesingle{}Carbon tax share\textquotesingle{}}\NormalTok{, }\StringTok{\textquotesingle{}EV Subsidy share\textquotesingle{}}\NormalTok{))}
\end{Highlighting}
\end{Shaded}

\begin{table}

\caption{\label{tab:3.D.print}Illustration of Policy Effects for Three Vehicles}
\centering
\begin{tabular}[t]{llrrrr}
\toprule
Make & Model & Lifetime CO2 costs & Baseline share & Carbon tax share & EV Subsidy share\\
\midrule
tesla & model y & 3237.018 & 0.0009334 & 0.0008619 & 0.0011263\\
honda & civic & 9192.808 & 0.0005064 & 0.0004027 & 0.0005062\\
toyota & tacoma & 16776.875 & 0.0004499 & 0.0002957 & 0.0004497\\
\bottomrule
\end{tabular}
\end{table}

\hypertarget{e}{%
\subsubsection{3.E}\label{e}}

\(\Delta PS = 0\) by assumption. I use the following formulas for the
other components of surplus change (from the slides, where 1 denotes the
post-policy value and 0 denotes the pre-policy value).

\[
\Delta CS = \frac{1}{-\eta} \cdot \left(\log\left(1 + \sum_j e^{V_j^1}\right) - \log \left(1 + \sum_j e^{V_j^0}\right) \right) 
\] \[
\Delta G = \sum_j t_j s_j^1 
\] \[
\Delta E = \sum_j -\phi_j(s_j^1 - s_j^0)
\] \[
\Delta TS = \Delta CS + \Delta G + \Delta E
\]

And cost/ton of carbon abated is
\((\Delta CS + \Delta G)/\Delta Carbon\). In cases where
\(\Delta Carbon\) is negative, I simply report ``undefined''. I present
results in Table 3.

\begin{Shaded}
\begin{Highlighting}[]
\CommentTok{\# Tax}
\NormalTok{Del\_CS\_1 }\OtherTok{=} \SpecialCharTok{{-}} \DecValTok{1000}\SpecialCharTok{/}\NormalTok{logit\_out}\SpecialCharTok{$}\NormalTok{coefficients[}\StringTok{\textquotesingle{}price\textquotesingle{}}\NormalTok{] }\SpecialCharTok{*}\NormalTok{ (}
  \FunctionTok{log}\NormalTok{(}\DecValTok{1} \SpecialCharTok{+} \FunctionTok{sum}\NormalTok{(}\FunctionTok{exp}\NormalTok{(df\_EV}\SpecialCharTok{$}\NormalTok{V1))) }\SpecialCharTok{{-}} \FunctionTok{log}\NormalTok{(}\DecValTok{1} \SpecialCharTok{+} \FunctionTok{sum}\NormalTok{(}\FunctionTok{exp}\NormalTok{(df\_EV}\SpecialCharTok{$}\NormalTok{V0)))}
\NormalTok{) }\CommentTok{\# note the unit cnversion (multiply by 1000)}

\NormalTok{Del\_G\_1 }\OtherTok{=} \FunctionTok{sum}\NormalTok{(df\_EV}\SpecialCharTok{$}\NormalTok{s1 }\SpecialCharTok{*}\NormalTok{ df\_EV}\SpecialCharTok{$}\NormalTok{co2cost)}
\NormalTok{Del\_E\_1 }\OtherTok{=} \FunctionTok{sum}\NormalTok{(}\SpecialCharTok{{-}}\NormalTok{df\_EV}\SpecialCharTok{$}\NormalTok{co2cost }\SpecialCharTok{*}\NormalTok{ (df\_EV}\SpecialCharTok{$}\NormalTok{s1 }\SpecialCharTok{{-}}\NormalTok{ df\_EV}\SpecialCharTok{$}\NormalTok{share))}
\NormalTok{Del\_TS\_1 }\OtherTok{=}\NormalTok{ Del\_CS\_1 }\SpecialCharTok{+}\NormalTok{ Del\_G\_1 }\SpecialCharTok{+}\NormalTok{ Del\_E\_1}

\NormalTok{Del\_carbon\_1 }\OtherTok{=} \FunctionTok{sum}\NormalTok{(df\_EV}\SpecialCharTok{$}\NormalTok{co2 }\SpecialCharTok{*}\NormalTok{ (df\_EV}\SpecialCharTok{$}\NormalTok{s1 }\SpecialCharTok{{-}}\NormalTok{ df\_EV}\SpecialCharTok{$}\NormalTok{share))}
\NormalTok{CE\_1 }\OtherTok{=}\NormalTok{ (Del\_CS\_1 }\SpecialCharTok{+}\NormalTok{ Del\_G\_1)}\SpecialCharTok{/}\NormalTok{Del\_carbon\_1}

\CommentTok{\# Subsidy}
\NormalTok{Del\_CS\_2 }\OtherTok{=} \SpecialCharTok{{-}} \DecValTok{1000}\SpecialCharTok{/}\NormalTok{logit\_out}\SpecialCharTok{$}\NormalTok{coefficients[}\StringTok{\textquotesingle{}price\textquotesingle{}}\NormalTok{] }\SpecialCharTok{*}\NormalTok{ (}
  \FunctionTok{log}\NormalTok{(}\DecValTok{1} \SpecialCharTok{+} \FunctionTok{sum}\NormalTok{(}\FunctionTok{exp}\NormalTok{(df\_EV}\SpecialCharTok{$}\NormalTok{V2))) }\SpecialCharTok{{-}} \FunctionTok{log}\NormalTok{(}\DecValTok{1} \SpecialCharTok{+} \FunctionTok{sum}\NormalTok{(}\FunctionTok{exp}\NormalTok{(df\_EV}\SpecialCharTok{$}\NormalTok{V0)))}
\NormalTok{)}

\NormalTok{Del\_G\_2 }\OtherTok{=} \FunctionTok{sum}\NormalTok{(df\_EV}\SpecialCharTok{$}\NormalTok{s2 }\SpecialCharTok{*}\NormalTok{ df\_EV}\SpecialCharTok{$}\NormalTok{eligible }\SpecialCharTok{*} \SpecialCharTok{{-}}\DecValTok{7500}\NormalTok{)}
\NormalTok{Del\_E\_2 }\OtherTok{=} \FunctionTok{sum}\NormalTok{(}\SpecialCharTok{{-}}\NormalTok{df\_EV}\SpecialCharTok{$}\NormalTok{co2cost }\SpecialCharTok{*}\NormalTok{ (df\_EV}\SpecialCharTok{$}\NormalTok{s2 }\SpecialCharTok{{-}}\NormalTok{ df\_EV}\SpecialCharTok{$}\NormalTok{share))}
\NormalTok{Del\_TS\_2 }\OtherTok{=}\NormalTok{ Del\_CS\_2 }\SpecialCharTok{+}\NormalTok{ Del\_G\_2 }\SpecialCharTok{+}\NormalTok{ Del\_E\_2}

\NormalTok{Del\_carbon\_2 }\OtherTok{=} \FunctionTok{sum}\NormalTok{(df\_EV}\SpecialCharTok{$}\NormalTok{co2 }\SpecialCharTok{*}\NormalTok{ (df\_EV}\SpecialCharTok{$}\NormalTok{s2 }\SpecialCharTok{{-}}\NormalTok{ df\_EV}\SpecialCharTok{$}\NormalTok{share))}
\NormalTok{CE\_2 }\OtherTok{=}\NormalTok{ (Del\_CS\_2 }\SpecialCharTok{+}\NormalTok{ Del\_G\_2)}\SpecialCharTok{/}\NormalTok{Del\_carbon\_2}
\NormalTok{CE\_2 }\OtherTok{=} \StringTok{\textquotesingle{}Undefined\textquotesingle{}} \CommentTok{\# ad{-}hoc change since this is \textless{} 0...}

\NormalTok{Surplus\_Tab }\OtherTok{=} \FunctionTok{data.frame}\NormalTok{(}
  \StringTok{\textquotesingle{}Label\textquotesingle{}} \OtherTok{=} \FunctionTok{c}\NormalTok{(}\StringTok{\textquotesingle{}Change in CS ($/consumer{-}year)\textquotesingle{}}\NormalTok{, }\StringTok{\textquotesingle{}Change in PS ($/consumer{-}year)\textquotesingle{}}\NormalTok{, }\StringTok{\textquotesingle{}Change in G ($/consumer{-}year)\textquotesingle{}}\NormalTok{, }\StringTok{\textquotesingle{}Change in E ($/consumer{-}year)\textquotesingle{}}\NormalTok{, }\StringTok{\textquotesingle{}Change in TS ($/consumer{-}year)\textquotesingle{}}\NormalTok{, }\StringTok{\textquotesingle{}Cost{-}effectiveness ($/ton of CO2 abated)\textquotesingle{}}\NormalTok{),}
  \StringTok{\textquotesingle{}CO2 Tax\textquotesingle{}} \OtherTok{=} \FunctionTok{round}\NormalTok{(}\FunctionTok{c}\NormalTok{(Del\_CS\_1, }\DecValTok{0}\NormalTok{, Del\_G\_1, Del\_E\_1, Del\_TS\_1, CE\_1),}\DecValTok{0}\NormalTok{),}
  \StringTok{\textquotesingle{}EV Subsidy\textquotesingle{}} \OtherTok{=} \FunctionTok{c}\NormalTok{(}\FunctionTok{round}\NormalTok{(}\FunctionTok{c}\NormalTok{(Del\_CS\_2, }\DecValTok{0}\NormalTok{, Del\_G\_2, Del\_E\_2, Del\_TS\_2),}\DecValTok{0}\NormalTok{),CE\_2)}
\NormalTok{)}

\NormalTok{kableExtra}\SpecialCharTok{::}\FunctionTok{kable}\NormalTok{(Surplus\_Tab, }\AttributeTok{booktabs =}\NormalTok{ T, }\AttributeTok{caption =} \StringTok{\textquotesingle{}Effect of Policies on Surplus\textquotesingle{}}\NormalTok{)}
\end{Highlighting}
\end{Shaded}

\begin{table}

\caption{\label{tab:3.E}Effect of Policies on Surplus}
\centering
\begin{tabular}[t]{lrl}
\toprule
Label & CO2.Tax & EV.Subsidy\\
\midrule
Change in CS (\$/consumer-year) & -64 & 17\\
Change in PS (\$/consumer-year) & 0 & 0\\
Change in G (\$/consumer-year) & 55 & -19\\
Change in E (\$/consumer-year) & 20 & -1\\
Change in TS (\$/consumer-year) & 10 & -3\\
\addlinespace
Cost-effectiveness (\$/ton of CO2 abated) & 90 & Undefined\\
\bottomrule
\end{tabular}
\end{table}

\hypertarget{part-4-summary}{%
\section{Part 4: Summary}\label{part-4-summary}}

Conceptually, a carbon tax will have a more positive impact on total
surplus than an EV subsidy for at least two reasons.

First, a carbon tax induces socially beneficial substitution away from
more carbon-intensive gas-using vehicles and towards less
carbon-intensive gas-using vehicles (e.g, from the Tacoma to the Civic).
The histogram in figure 4 demonstrates that there is significant
heterogeneity in externality costs even across gas-using vehicles, so
these gains may be important.

Second, the carbon tax discourages EV consumption relative to the
outside option of not buying a car, while the EV subsidy encourages EV
consumption. Since EVs do use carbon, this effect is undesirable. The
subsidy is good to the extent that it encourages substitution from
carbon-intensive cars to EVs but bad to the extent that it encourages
substitution from not driving to EVs. The overall change in externality
(and therefore the overall change in total surplus) is negative in this
example because the bad type of substitution outweighs the good type of
substitution.

The above is a sound policy argument in favor of the carbon tax. Howver,
the distributional consequences of the two policies are drastically
different. The carbon tax hurts consumers and increases government
revenue, but the EV subsidy benefits consumers and decreases government
revenue. To the extent that government revenue is redistributed to
people with high marginal values of money (e.g, through welfare payments
or social insurance), this may constitute a distributional argument in
favor of the carbon tax. If you believe government funds are less
valuable, this argument has less bite.

In practice, this difference may explain why EV subsidies are more
politically feasible than carbon taxes. The government costs and
benefits are diffuse and consumers may not notice the marginal impact of
the extra tax revenue, while consumers do notice extra costs every time
they go to the gas pump or a large discount on their electric vehicle
purchase.

We made many modelling assumptions here. I think the most important one
was that our demand estimation was unconfounded. It seems plausible that
unobservable quality would be correlated with price. If this was true,
then consumers are more elastic than our estimates indicate. Higher
elasticities amplify responses to taxes and subsidies, so this story
would mean our estimates of the surplus changes were all biased towards
zero.

\end{document}
